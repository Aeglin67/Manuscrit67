\chapter{Introduction}
\label{chap:intro}
\minitoc

Chapitre destin� � introduire les notions fondamentale ayant guid�es les travaux.

%------------------------------------------------------------------------Pr�sentation----------------------------------------------------------------------------------------

\section{Pr�sentation} 

\subsection{Contexte}

\subsection{Objectifs}

\subsection{Organisation du manuscrit}

%------------------------------------------------------------------------Anatomie et maturation c�r�brale----------------------------------------------------------------------------------------

\section{Anatomie et maturation c�r�brale}
Pr�sentation des diff�rents tissus c�r�braux et comment leur �volution se refl�te � l'IRM.
Livre : \cite{Rutherford:2001} avec une partie sur les pr�matur�s.

\subsection{Les diff�rents tissus c�r�braux}

\subsection{Maturation c�r�brale}

%------------------------------------------------------------------------IRM----------------------------------------------------------------------------------------

\section{L'imagerie par r�sonance magn�tique}
\label{intro:irm}

\subsection{Principe de l'IRM}

\subsection{Formation des images et contrastes}
Comment obtient-on une image T1, T2 ou densit� de protons.

\subsection{Caract�ristiques}
\label{intro:irm:carac}

Volume partiel, bruit, biais en intensit�.