\chapter{Introduction}
\label{chap:intro}
\minitoc

Chapitre destin� � introduire les notions fondamentale ayant guid�es les travaux.

%------------------------------------------------------------------------Pr�sentation----------------------------------------------------------------------------------------

\section{Pr�sentation} 

\subsection{Contexte}

\subsection{Objectifs}

\subsection{Organisation du manuscrit}

%------------------------------------------------------------------------Anatomie et maturation c�r�brale----------------------------------------------------------------------------------------

\section{Anatomie et maturation c�r�brale}
Pr�sentation des diff�rents tissus c�r�braux et comment leur �volution se refl�te � l'IRM.
Livre : \cite{Rutherford:2001} avec une partie sur les pr�matur�s.

\subsection{Les diff�rents tissus c�r�braux}

\subsection{Maturation c�r�brale}

\cite{Prayer:PedRad:2006} : description de la maturation c�r�brale avec acquisitions IRM

%------------------------------------------------------------------------IRM----------------------------------------------------------------------------------------

\section{L'imagerie par r�sonance magn�tique}
\label{intro:irm}

\subsection{Principe de l'IRM}

Livre en ligne : \cite{Hornak:Online:1996}.

Principe de la RMN : deux articles ind�pendant qui sont \cite{Bloch:PhyRev:1946} et \cite{Purcell:PhyRev:1946}.
Premi�res vraies images : \cite{Mansfield:JPC:1973} et \cite{Lauterbur:Nature:1973}.

\cite{Hubbard:SemPer:1999} : article d�crivant le protocole d'IRM pour les foetus

\subsection{Formation des images et contrastes}
Comment obtient-on une image T1, T2 ou densit� de protons.

\subsection{Caract�ristiques}
\label{intro:irm:carac}

\subsubsection{Le bruit}

\cite{Sijbers:TMI:1998} : article pr�cisant que le bruit rencontr� dans les IRM est de type ricien et peut �tre approxim� par une distribution gaussienne avec un haut SNR.

\subsubsection{Le volume partiel}

\subsubsection{Le biais en intensit�}

\subsubsection{Autres}

